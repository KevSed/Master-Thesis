\chapter{Summary}\label{ch:summary}
%
Paar Sachen gehören wohl eher in den outlook
\begin{itemize}
  \item there is an open data FACT sample, ready to be used for analysis
  \item structure and tools to do so with PhotonStream data have been set up
  \item PhotonStream data enables new methods and features for analyses
  \item PhotonStream data contains more photons and therefore seems to reconstruct air-showers very efficient and closer to reality. but still similar to LP data format
  \item with DBSCAN cleaning PhotonStream air-showers have larger spatial distributions along the pixels, which may need to be accounted for
  \item on PhotonStream data all the classical features plus a large number of unprecedented features can be engineered
  \item the main classical features on PhotonStream data have significant data MC mismatches
  \item new time features can be implemented, but first hints on even stronger mismatches on time information in MC simulations
  \item the topology of PhotonStream air-showers affects the orientation of the calculated shower axes strongly and causes the origin reconstruction to become blurred
  \item a high performance origin reconstruction with the standard analysis very hard, \texttt{disp} regression not working well even on MC data
  \item very high performance of the gamma hadron separation on MC data, better than compared FACT-Tools analysis
  \item high performance of energy estimation on MC simulations, also better than FACT-Tools
  \item DBSCAN cleaning seems to work best on PhotonStream data
  \item other cleanings on threedimensional space possible
  \item DBSCAN as implemented not really possible to be improved, rather optimize cuts in study since data mc mismatches have to be accounted for
  \item classical pixel based threshold cleaning yields poor performance, threedimensional cleaning way to go
\end{itemize}
