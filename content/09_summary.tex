\chapter{Conclusion and Outlook}\label{ch:summary}
%
The PhotonStream offers a great number of new opportunities for future IACT
analyses, only a few of which have been investigated in this analysis.
In this work a very first analysis of the new IACT data representation
PhotonStream is performed. It exclusively uses openly accessible data from the
FACT open data sample~\cite{fact-data} and only open source software tools. The
necessary data handling and feature generation tools have been implemented in
the openly available \texttt{FeatureStream}~\cite{FeatureStream}. The extensive
investigations on the different characteristics of the new representation and
the implications for the analysis of PhotonStream data resulted in several
remarks and yield the basis for a new, dedicated PhotonStream analysis.

The PhotonStream opens new ways to represent IACT data and to parametrize air-
shower events for analysis. The single photon extraction results in a larger
number of reconstructed photons and enables a more efficient and exact way to
reconstruct air-shower events. The resulting images still suit the classical LP
representation very well. The new per photon timing information impacts every
analysis step. The density-based cleaning DBSCAN is a completely new way to
clean air-shower events of the background. The resulting data set is
characterized by brighter images with larger distributions along the spatial
coordinates, that require new thresholds for cleanings and special treatment of mismatches between simulations and observations. Generally, the PhotonStream data contains the
information neccessary to generate the classical Hillas features and additional
fetaures used in FACT analyses. Additionally, the new three-dimensional
representation offers a great number of new features to be engineered. The
classical features have been investigated in this work and show significant
mismatches to the respective simulations. The mismatches have a strong impact
on the analysis results and need to be accounted for by quality cuts or
improving the simulation quality.

The timing information within the PhotonStream shows even bigger mismatches and
therefore needs better simulation quality and optimized analysis methods to
boost the analysis performance. The different topology of cleaned air-shower
events strongly affects the used machine learning techniques for the origin
reconstruction. The orientation of air-showers shows significant mismatches,
making the reconstrcution error very large and the reconstructed signal source
blurred. The disp-method used for the origin reconstruction does not perform
very well, even on the MC simulations it is trained on. The separation of gamma
air-showers and hadronic air-showers is working very well on MC simulations.
The performance measures yield better results than the comparative FACT-Tools
analysis. The energy estimation of the gamma air-showers yields a significantly
better performance than the FACT-Tools analysis. Both these results indicate
that the cleaned air-shower clusters show a better agreement with the true air-
showers and therefore can better be reconstructed. The used unsupervised
clustering algorithm DBSCAN shows much better results on PhotonStream data than
the classical threshold based cleaning. An improvement of the mismatches
between observations and MC simulations via this algorithm's hyperparameters
could not be achieved. It seems that the data needs to be cleaned of noises and
the different topologies of observed data and MC simulations need to be
understood better to improve. To achieve this dedicated studies for quality
cuts and the noises in the data samples are needed. However, since the two-dimensional cleaning on the shower pixels has a poor performance the three-dimensional space is the most promising space for event cleaning.

The PhotonStream data representation shows very good tendencies for energy
reconstruction and gamma hadron separation and therefore promising
opportunities for the field of imaging air Cherenkov telescopes. It is not only
a very intuitive data representation but also one suited very well for
analysing air-shower images, as is done by FACT, representing a great new development in IACT data.
