\chapter{Introduction}
\nocite{biblatex, siunitx, Hunter:2007}%
%
\begin{aquote}{\textit{Albert Einstein}}
\textsc{One cannot help but be in awe when he contemplates the mysteries of eternity, of life, of the marvelous structure of reality.}
\end{aquote}
In physics, the search for a fundamental understanding of reality is ubiquitous. The understanding of the universe we live in and the rules of nature has
led mankind to look for the things beyond Earth. To understand what these rules
are, on Earth and in the universe, is the goal of astrophysics. The electromagnetic radiation reaching the Earth from space has opened the window to learn
about the universe beyond our planet. The discovery of this extraterrestrial
radiation by Victor Hess during his balloon flights around \num{100} years ago in \num{1912}~\cite{Hess} marks the beginning of the observation of this radiation.
Today we know that the electromagnetic radiation spans a very large range of
frequencies and reaches up to the highest energies ever measured \cite{source}.
The different energy ranges of cosmic radiation caused the development of a
variety of different telescopes, specialized on certain frequency bands. On
ground, only visible light and radio waves from space can be observed directly,
due to the absorption of Earth's atmosphere. Therefore, large optical and radio
telescopes have been built to observe stars, galaxies and other objects from
Earth. Of course, the atmosphere still has an impact on the observations, so
there are optical telescopes in space as well, like the Hubble Space
Telescope~\cite{hubble}, observing infra-red to ultra-violett light. However,
some telescopes take advantage of the atmosphere's impact, by using it as a
giant detector volume. Among these telescopes are the Imaging Air Cherenkov
Telescopes (IACT), observing cosmic rays via their interaction products. This
work is based on the observations of such a telescope, described in
\autoref{ch:fact}. The development and advancement of new telescopes continues
today, as the currently constructed Cherenkov Telescope Array CTA~\cite{cta}
will yield unprecedented accuracy and sensitivity among IACTs.
Among the goals of astrophysics are not only the understanding of the
structures of the universe, but also the very elementary interactions of
particles within the sources of this radiation. The very first hints for the
existance and characteristics of dark matter~\cite{zwicky}, a supposed
elementary particle for instance originate from the field of astrophysics.
This work focuses on the rather technical aspect of a new data representation
and its implications for IACT analyses, by analyzing a data set of this new data
format for the first time.
