% \chapter{Conclusion and Outlook}
% %

The promising results of the analysis steps performed in this work implicate
several working fields for future analyses on PhotonStream data. Almost
all steps of the analysis of IACT data can be adapted to the new data
representation and optimized to the new features. Especially when considering
that the compared FACT-Tools analysis is highly specialized and based on years
of experience, the potential of the PhotonStream is very promising.

An outlook for a few example future improvements of the analysis of FACT
PhotonStream data are:

\begin{itemize}
  \item Improving the used metric defining the three-dimensional spacetime of the PhotonStream. The metric has a strong impact on the cluster topology and might help reduce mismatches.
  \item The new timing information of the PhotonStream might be used to generate new features to boost the performance of gamma hadron separation and origin reconstruction. The mismatches of observations and MC simulations need to be investigated alongside the generation of new features.
  \item Dedicated studies to find the optimal quality cuts for this data sample and improving the compliance of observations and MC simulations will improve the performance and might make it comparable to the classical analyses.
  \item Engineering new features will contribute to better performance and might improve IACT analyses beyond the currently possible performance. They may also allow for new investigations e.\,g. of the time structures of air-showers.
  \item The three-dimensional point cloud allows for new cleaning algorithms to be used, which may improve the quality of the reconstructed air-showers even more.
\end{itemize}

The PhotonStream data representation offers a great number of possible future studies not only for the FACT collaboration, but the whole IACT community. The
understanding of the data within this new representation and an improvement of simulations is crucial, to achieve better performance. This work is the first step to achieve a better understanding of the data and build the basis for a new way of analyzing IACT data.
