\chapter{Imaging Air Cherenkov Astronomy}

The atmosphere of earth is continously penetrated by radiation from different
sources within our universe. This cosmic radiation is made up of different types
of particles, interacting with the atmosphere in various ways. There are
charged protons and the uncharged neutrinos and photons (gamma rays). Neutrinos
are uncharged, very light fermions, that interact very weakly and are not
detectable by optical telescopes at all. Protons make up the largest number of
particles reaching earths atmosphere. Due to their electric charge, they are
deflected by magnetic fields and therefore lose information on their origin on
the way to earth, making them unsuitable for Cherenkov astronomy.

Cosmic gamma-rays are photons with a very high energy, originating from bright
sources such as active galactic nuclei or nebulae. When such particles intersect
with earth's atmosphere, they move faster than light within that atmosphere due
to their high energy. Particles moving at such speeds through a medium cause,
among the creation of other particles, the emission of bluish photons, the
Cherenkov-light. Cherenkov-light is emitted directly from the moving particle
within a specific angle towards the direction of movement of that primary
particle.
%
\begin{equation}
    \cos(\vartheta) = \frac{1}{n\beta}
    \label{eq:angle_cherenkov}
\end{equation}
%
As \autoref{eq:angle_cherenkov} shows, this angle depends on the index of
refraction $n$ of the medium and the particle's velocity. Due to this emission
angle the light traverses the medium in a cone-shape, when being described from
earth's point of view. By the time it is reaching the ground it thus
illuminates an elliptical area of about $\SI{200}{\meter}$ diameter, depending
on the height of interaction and the primary particle's energy.
This already implicates that the light, although a secondary product of the
cosmic gamma-ray, can be used to reconstruct physical properties of
said gamma-ray. To do so, the flashes of the Cherenkov-light need to
be captured by cameras capable of filming very short time scales (about
$\SI{e-9}{\second}$).

Imaging Air Cherenkov Telescopes (IACTs) are using videos of this light, to
reconstruct properties of the incident cosmic radiation, by analyzing
properties of the measured pictures. By doing so, the atmosphere is used as a very large Cherenkov-detector material.

The three properties of interest are:
\begin{description}[labelsep=3em, align=right]
  \item[source position]{the position of the source of the primary particle on the sky}
  \item[particle type]{the distinction between cosmic gamma rays and other particles like protons or secondary particles like muons}
  \item[particle energy]{the energy of the primary gamma-ray}
\end{description}
