\chapter{Appendix}
\addtocontents{toc}{\protect\setcounter{tocdepth}{-1}}
%
\section{Plots}
\begin{table}
  \centering%
  \begin{tabular}{l
                  c
                  c}
      \toprule
      {}    & Anzahl der Filter in den Dichtelagen  & Struktur der Dropout Lagen      \\
      \midrule
      Modell 0    & (1024, 512, 128, 64, 32)  & (0.5, 0.4, 0.4, 0.3, 0.2) \\
      Modell 1    & (1024, 512, 256, 128, 64, 32, 16)  & (0.5, 0.4, 0.4, 0.4, 0.2, 0.2, 0.1) \\
      Modell 2    & (512, 256, 128, 64, 32, 16)  & (0.4, 0.4, 0.3, 0.3, 0.2, 0.1) \\
      Modell 3    & (1024, 256, 64, 16)  & (0.6, 0.4, 0.2, 0.1) \\
      Modell 4    & (512, 128, 32)  & (0.5, 0.3, 0.1) \\
      \bottomrule
  \end{tabular}
  \caption{Getestete Grundstrukturen für die Netzarchitekturen der alternativen Methode. Das $n$-te Element der Tupel beschreibt jeweils die Filtergröße der $n$-ten Dichtelage. Selbiges gilt für die Dropout Lagen. Es folgt auf jede Dichtelage (abgesehen von der letzten Lage) eine Dropout Lage.}
  \label{tab:grid}
\end{table}
%




\addtocontents{toc}{\protect\setcounter{tocdepth}{0}}
