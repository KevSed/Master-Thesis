\thispagestyle{plain}
\section*{\color{tugreen}Abstract}
%
In imaging air Cherenkov astronomy, the cosmic radiation is observed via the
secondary particles it produces in Earth's atmosphere. These high-energy secondary
particles cause the emission of Cherenkov light, which is measured by the
Cherenkov telescopes' optical cameras. From this light, information on the
cosmic radiation can be derived, to learn about the characteristics of the
cosmic sources. The data, Cherenkov telescopes produce, consists of
properties derived from the time series of measured voltages. These properties
strongly depend on the detector's readout hardware. This work focuses on a new
data representation, developed by Sebastian Mueller: the PhotonStream. It
consists of single reconstructed photons and opens new possibilities for
analyses throughout the field of Cherenkov astronomy. In this work, a first analysis of this new data representation is
performed. To do so, openly accessible data of the First G-APD Cherenkov
telescope is used. Alongside the results of origin reconstruction, gamma hadron
separation and energy reconstruction, detailed investigations on the
PhotonStream's properties and differences to the classical approach are
presented.

\section*{\color{tugreen}Kurzfassung}
%
\begin{german}
Die bildgebende atmosphärische Cherenkov Astronomie beschäftigt sich mit der
Untersuchung kosmischer Strahlung über die Sekundärprodukte dieser. Beim
Durchqueren der Atmosphäre entstehen hochenergetische Sekundärteilchen und die
Emission von Cherenkov Licht wird angeregt. Dieses Licht wird von den optischen
Kameras der Cherenkov Teleskope gemessen, um Informationen über die
Eigenschaften dieser Strahlung und ihrer kosmischen Quellen zu sammeln. Die
Daten, die dabei entstehen, stammen aus Spannungszeitreihen der Kamerasensoren
und werden daher stark von den Eigenheiten dieser beeinflusst. Diese Arbeit
beschäftigt sich mit einer neuen Datenrepräsentation, welche von Sebastian
Mueller entwickelt wurde: dem PhotonStream. Diese Repräsentation nutzt die
Möglichkeit, einzelne Photonen aufzulösen, um einen Datensatz aus diesen zu
erstellen, welcher gleichzeitig von der direkten Sensorantwort entkoppelt ist.
Durch die Darstellung von einzelnen Photonen eröffnen sich neue Möglichkeiten
für die Analyse der Daten und das gesamte Feld der Cherenkov Astronomie.
Diese Arbeit stellt eine erste Analyse dieser Daten aus dem
öffentlichen Datensatz des \textit{First G-APD Cherenkov telescope} dar. Neben
den Analyseergebnissen der Quellrekonstruktion, der Gamma Hadron Separation,
sowie der Energieschätzung, werden detaillierte Untersuchungen der
Eigenschaften und Unterschiede dieser neuen Datenrepräsentation präsentiert.
\end{german}
